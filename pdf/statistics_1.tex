\documentclass[a4paper,twocolumn,dvipdfmx]{jsarticle}
%b4paperは用紙のサイズ,b4のところをa4に変更などすればサイズも変わる,landscapleは用紙横向き%
\usepackage[top=20truemm,bottom=20truemm,left=13truemm,right=13truemm]{geometry}
%余白の設定,単位はcm,mmでも可%


%様々なパッケージたち,パッケージ同士で衝突することもあるので注意が必要%
\usepackage{okumacro}
\usepackage{fancyhdr}
\usepackage{lastpage}
\usepackage{mathrsfs}
\usepackage[dvipdfmx,hidelinks]{hyperref}
\usepackage{pxjahyper}
\usepackage{amsmath}
\usepackage{amsfonts}
\usepackage{ascmac}
\usepackage{color}
\usepackage{amssymb}
\usepackage[dvipdfmx]{graphicx}
\usepackage[dvipdfmx]{color}
\usepackage{graphics}
\usepackage{tikz}
\usepackage{tikz-cd}
\usepackage{bm}
\usepackage{bbm}
\usepackage{picture}
\usepackage{fancybox}
\usepackage[deluxe]{otf}
\usepackage{stmaryrd}
\usepackage{hhline}
\usepackage{longtable}


%定理環境%
\usepackage{amsthm}
\usepackage[xcolor]{mdframed}
\usepackage{silence}\WarningFilter{mdframed}{You got a bad break}
\let\oldtheorem=\newtheorem
\mdfdefinestyle{leftbar}{topline=false,rightline=false,bottomline=false,innertopmargin=0pt,linewidth=1pt,linecolor=black} \newcommand{\NewTheorem}[2][linecolor=black]{\surroundwithmdframed[style=leftbar,#1]{#2}\oldtheorem{#2}}
\newcommand{\NewTheoremStar}[2][linecolor=black]{\surroundwithmdframed[style=leftbar,#1]{#2}\oldtheorem*{#2}}
\makeatletter\renewcommand{\newtheorem}{\@ifstar{\NewTheoremStar}{\NewTheorem}}\makeatother
\newtheoremstyle{mydefinition}{\topsep}{\topsep}{\rmfamily}{0pt}{\sffamily\gtfamily\bfseries}{.}{.5em}
{\thmname{#1}\thmnumber{ #2}\thmnote{ (#3)}\rmfamily}
\theoremstyle{mydefinition}

\newtheorem{dfn}{定義}
\newtheorem[leftline=false,backgroundcolor=black!30!white]{thm}[dfn]{定理}
\newtheorem{prop}[dfn]{命題}
\newtheorem{lem}[dfn]{補題}
\newtheorem{cor}[dfn]{系}
\newtheorem{fact}{事実}
\newtheorem{exam}{例題}
\newtheorem{axi}{公理}
\newtheorem{prob}{問題}
\newtheorem[leftline=false]{prac}{練習}

%番号付けの設定,arabicがアラビア数字(1,2,3,4,...),alphはアルファベット(a,b,c,d,...),romanはローマ数字(i,ii,iii,iv,...)%
\renewcommand{\labelenumii}{\textbf{\textsf{\arabic{enumii}}.}} %一番外の番号(例えば,大問みたいな感じ)%
\renewcommand{\labelenumi}{(\arabic{enumi})}%二番目の番号(例えば,大問のつぎの数字)%
\renewcommand{\labelenumiii}{(\alph{enumiii})}%三番目の番号(例えば,小問みたいな感じ)%
\renewcommand{\thefootnote}{\arabic{footnote})}%注釈の番号%


%いちいち,打つのがめんどくさかったり,覚えにくい既存のコマンドを,自分用に設定%
%以下,\newcommand{\*}{\***}タイプ:本来,"\***"と入力すべきところを"\*"と省略%
\newcommand{\s}{\mathfrak{S}}
\newcommand{\bs}{\textbackslash}
\newcommand{\txhat}{\textasciicircum}
\newcommand{\vsp}{\vspace{.25zw}}

%以下,\newcommand{\**}[n]{\***{#i}\*{#j}.../****{#k}} (n,i,j,kは何かしらの数値;i,j,kはn以下の数値で相異なる)タイプ:nは引数,"\**{#1}{#2}...{#n}"と入力すれば,"\***{#i}\*{#j}.../****{#k}"が実行される%
\newcommand{\lfrac}[2]{\left(\frac{#1}{#2}\right)}
\newcommand{\Vect}[1]{\overrightarrow{\mathrm{#1}}}
\newcommand{\vect}[1]{\overrightarrow{#1}}
\newcommand{\abs}[1]{\left|#1\right|}

%既存の環境を自分でオリジナルな環境に設定,最近使い始めたのでちゃんとは理解していない%
\newenvironment{smallpmatrix}{\left(\begin{smallmatrix}}{\end{smallmatrix}\right)}
\newenvironment{smallvmatrix}{\left|\begin{smallmatrix}}{\end{smallmatrix}\right|}
\newenvironment{smallbmatrix}{\left[\begin{smallmatrix}}{\end{smallmatrix}\right]}

\renewcommand{\footrulewidth}{0.4pt}%下の線%
\setlength{\columnseprule}{0.4pt}%真ん中の線%
\pagestyle{fancy}
\lhead{数学探求I,I\hspace{-0.1em}I\quad データの分析 補足}%枠外右上の文字%
\chead{}
\cfoot{(\thepage/\pageref{LastPage})}
\lfoot{}%枠外左下の文字%
\rfoot{文責:中橋}%枠外右下の文字%
%\rhead{\underline{\qquad\qquad}\,年\ \underline{\qquad\qquad}\,組\ \underline{\qquad\qquad}\,番\quad 氏名\,\underline{\hspace{20zw}}}%枠外左上の文字%


\newcommand{\ans}[1]{{\color{white}\bf{#1}}}


\hypersetup{ pdftitle={},pdfauthor={Kentaro NAKAHASHI}}
\AtBeginDvi{\special{pdf:encrypt ownerpw(nk210512) userpw(r3mq1-2)length 128 perm 0}}

\allowdisplaybreaks
\begin{document}
\gtfamily
	\section*{\textbf{変量の変換}}
	\begin{thm}
	\gtfamily
		$a,\ b$を定数とする. 変量$x$のデータから$y=ax+b$によって新たな変量$y$のデータが得られるとき, $x,\ y$のデータの平均値を$\overline{x},\ \overline{y}$, 分散を${s_x}^2,\ {s_y}^2$, 標準偏差を$s_x,\ s_y$とすると, 次が成り立つ:\[\overline{y}=a\overline{x}+b,\quad {s_y}^2=a^2{s_x}^2,\quad s_y=\abs{a}{s_x}_.\]
	\end{thm}
	
	\noindent\textbf{証明.}
	
	\noindent 変量$x$のデータを\[x_1,\ x_2,\ \ldots,\ x_n\]とすると, 変量$y$のデータ$y_1,\ y_2,\ \ldots,y_n$はそれぞれ\[ax_1+b,\ ax_2+b,\ \ldots,\ ax_n+b\]となる. このとき, $y$の平均$\overline{y}$は
	\begin{align*}
		\overline{y}\hspace{.5zw}&=\hspace{.5zw}\dfrac{1}{n}\bigg(y_1+y_2+\cdots+y_n\bigg)\\[1zw]
		                 &=\hspace{.5zw}\dfrac{1}{n}\bigg\{(ax_1+b)+(ax_2+b)+\cdots+(ax_n+b)\bigg\}\\[1zw]
		                 &=\hspace{.5zw}\dfrac{1}{n}\bigg\{a(x_1+x_2+\cdots+x_n)+nb\bigg\}\\[1zw]
		                 &=\hspace{.5zw}a\cdot\dfrac{1}{n}\bigg(x_1+x_2+\cdots+x_n\bigg)+b\\[1zw]
		                 &=\hspace{.5zw}a\overline{x}+b
	\end{align*}
	よって, $\overline{y}=a\overline{x}+b.$\\
	
	\noindent 次に, $y$の分散${s_y}^2$は
	\begin{align*}
		{s_y}^2\hspace{.5zw}&=\hspace{.5zw}\dfrac{1}{n}\bigg\{(y_1-\overline{y})^2+(y_2-\overline{y})^2+\cdots+(y_n-\overline{y})^2\bigg\}\\[1zw]
		&=\hspace{.5zw}\dfrac{1}{n}\sum_{k=1}^n\big\{(ax_k+b)-(a\overline{x}+b)\big\}^2\\[1zw]
		&=\hspace{.5zw}\dfrac{1}{n}\sum_{k=1}^n\big\{a(x_k-\overline{x})\big\}^2\\[1zw]
		&=\hspace{.5zw}a^2\cdot\dfrac{1}{n}\bigg\{(x_1-\overline{x})^2+(x_2-\overline{x})^2+\cdots+(x_n-\overline{x})^2\bigg\}\\[1zw]
		&=\hspace{.5zw}a^2{s_x}^2
	\end{align*}
	よって, ${s_y}^2=a^2{s_x}^2.$\\
	
	\noindent 最後に, $y$の標準偏差$s_y$は$s_y=\sqrt{{s_y}^2}=\sqrt{a^2{s_x}^2}=\abs{a}{s_x}$となる. \hfill$\qed$\\
	
	
	
	\noindent\textbf{補足:} 一般に実数$x$に対して, $\sqrt{x^2}=\abs{x}$である. $\sqrt{x^2}=x$でない理由は...例えば, $2=\sqrt{4}=\sqrt{(-2)^2}.$\\

	
	\begin{shadebox}
		\begin{prac}[教科書p.175]
			ある都市の日ごとの最高気温を摂氏度$(^\circ\mathrm{C})$で計測し, 20日分のデータを得た. その平均値は$15.0\,^\circ\mathrm{C}$, 分散は$9.0$であった. このデータを華氏度$(^\circ\mathrm{F})$に変更したときの, 平均値, 分散, 標準偏差を求めよ. ただし, 摂氏度が$x\,^\circ\mathrm{C}$のときの華氏度を$y\,^\circ\mathrm{F}$とすると, $x$と$y$には次の関係がある.\[y=1.8x+32\]
		\end{prac}
		
		
	\end{shadebox}

	
	\newpage
	\subsection*{\textbf{分散と平均値の関係式}}\vsp\vsp\vsp\vsp
	
	\begin{thm}
	\gtfamily
	変量$u$に対して次の等式が成り立つ.
		\begin{align*}
			{s_u}^2\quad&=\quad\overline{u^2}-(\overline{u})^2\\
			                     &=\quad(u^2\text{の平均値})-(u\text{の平均値})^2.
		\end{align*}
	\end{thm}
	
	\noindent\textbf{証明.}
	
	\noindent 変量$u$のデータを\[u_1,\ u_2,\ \ldots,\ u_n\]とする. このとき, $u$の分散${s_u}^2$は
	\[{s_u}^2=\dfrac{1}{n}\bigg\{(u_1-\overline{u})^2+(u_2-\overline{u})^2+\cdots+(u_n-\overline{u})^2\bigg\}\]で与えられるから, 右辺を式変形すると
	\begin{align*}
	{s_u}^2\hspace{.5zw}&=\hspace{.5zw}\dfrac{1}{n}\bigg\{(u_1-\overline{u})^2+(u_2-\overline{u})^2+\cdots+(u_n-\overline{u})^2\bigg\}\\[1zw]
	&=\hspace{.5zw}\dfrac{1}{n}\left\{\sum_{k=1}^n{u_k}^2-2\overline{u}\sum_{k=1}^nu_k+n(\overline{u})^2\right\}\\[1zw]
	&=\hspace{.5zw}\underbrace{\dfrac{1}{n}\sum_{k=1}^n{u_k}^2}_{=\overline{u^2}}-2\overline{u}\cdot\underbrace{\dfrac{1}{n}\sum_{k=1}^nu_k}_{\large=\overline{u}}+(\overline{u})^2\\[1zw]
	&=\hspace{.5zw}\overline{u^2}-(\overline{u})^2
	\end{align*}
	ゆえに等式が成り立つ.\hfill$\qed$\\
	
	\noindent ※スペースの関係で足し算$(+\cdots+)$を省略して$\sum$(シグマ)の記号を使っている. 念のため説明しておくと \[\sum_{k=1}^na_k=a_1+a_2+\cdots+a_n\]の意味である. 詳しくは数学\textsf{B}の数列の単元参照.
	
	\newpage
	
	\begin{shadebox}
		\begin{prac}[教科書p.175]
			変量$x$のデータが次のように与えられている.\[750,\ 740,\ 720,\ 770,\ 750,\ 740\]いま, $c=10,\ x_0=740,\ u=\dfrac{x-x_0}{c}$として新しい変量$u$を作る.\footnotemark
			\begin{enumerate}
				\item 変量$u$のデータの平均値と標準偏差を求めよ.
				\item 変量$x$のデータの平均値と標準偏差を求めよ.
			\end{enumerate}
		\end{prac}
	\end{shadebox}
	
	
	\begin{center}
		\begin{tabular}{c||c|c|c|c|c|c|c}
		\hline
			$x$&750&740&720&770&750&740&$\setminus$\\\hline
			$u$&&&&&&&計\textcolor{white}{あああ}\\\hline
			$u^2$&&&&&&&計\textcolor{white}{あああ}\\\hline
		\end{tabular}

	\end{center}

	\footnotetext[1]{\gtfamily この$x_0$のことを\textbf{\gtfamily 仮平均}という.}
	
	
	
	
	
	
	
	
	
	
	
	
	
\end{document}